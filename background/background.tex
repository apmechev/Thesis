\chapter{Background Theory}

\label{ch:background}

\section{Introduction}

Astronomy is rapidly entering the big data regime with many large-scale sky surveys across the electromagnetic spectrum. This breadth of information is poised to expand the frontiers of Astronomy and Astrophysics and allow us to study and understand various phenomena in more detail. The longest wavelength of the spectrum accessible to Earth observatories lies in the Megahertz range, starting at 10 MHz, up to a Gigahertz, coresponding to wavelengths of 30 meters up to 30 cm. In Astronomy, this range is termed the low-frequency or meterwave regime. These wavelengths help uncover physical phenomena not accessible to projects in the X-ray, Visible, Infrared or microwave. In particular, long-wavelength radio can be used to study supermassive black holes, galaxy formation and evolution, magneto-hydrodynamics, solar physics, radio spectroscopy and many more. 

Photons provide us with rather few independent properties. Light provides us with the wavelength, the direction, the intensity  and the polarization of an astronomical source, as well as the change of those properties with time. Astronomers use these properties to better model distant sources, and confirm or reject astronomical theories. The accuracy of these models, or the rejection power of our observations are critically dependent on how accurately we can measure the four properties listed above. 

Astronomical observations in the long-wavelength regime have always been at the mercy of the diffraction limit, an effect that relates the wavelength of light, the diameter of the aperture and angular resolution obtained with that aperture. The angular resolution of a telescope determines how accurately the direction of an incoming photon is determined. Unfortunately the diffraction limit dictates that the angular resolution of a telescope with a fixed aperture decreases inversely proportional to the wavelength observed. For example, for a telescope at 100MHz to reach the same resolution as the 100-m diameter Effelsberg telescope at 10GHz, it would need a diameter of a 10 kilometers. Constructing, and operating a telescope of that size is currently outside our engineering capabilities, and thus low frequency astronomers have developed a method to synthesize a telescope aperture of an arbitrary size. 

Aperture synthesis is the practice of combining the signal of multiple antennas to produce data with the angular resolution of a much larger antenna. More specifically, the maximum angular resolution acheivable is proportional to the distance between your furthest two antennas. This technique has been used in a wide wavelength range, from the near- and mid-infrared (VLTI), submillimeter (ALMA) and radio wavelengths (VLA, GMRT, LWA). While this method is useful to increase the angular resolution of a telescope, it also requires significant post-observation computation in order to remove artifacts created by the aperture synthesis. In this work we aim to introduce LOFAR, the European Low Frequency Array, the data sizes and processing challenges that come with LOFAR data as well as our solutions to these challenges. We will conclude with the scientific results this work has led to, as well as suggestions for future large-scale astronomical projects

\subsection{LOFAR}
