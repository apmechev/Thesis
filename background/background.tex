\chapter{Background Theory}

\label{ch:background}

\section{Introduction}

Astronomy is rapidly entering the big data regime with many large-scale sky surveys across the electromagnetic spectrum. This breadth of information is poised to expand the frontiers of Astronomy and Astrophysics and allow us to study and understand various phenomena in more detail. The longest wavelength of the spectrum accessible to Earth observatories lies in the Megahertz range, starting at 10 MHz, up to a Gigahertz, coresponding to wavelengths of 30 meters up to 30 cm. In Astronomy, this range is termed the low-frequency or meterwave regime. These wavelengths help uncover physical phenomena not accessible to projects in the X-ray, Visible, Infrared or microwave. In particular, long-wavelength radio can be used to study supermassive black holes, galaxy formation and evolution, magneto-hydrodynamics, solar physics, radio spectroscopy and many more. 


