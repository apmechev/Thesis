Alexandar Mechev was born in Sofia, Bulgaria on the 21st of June, 1989. As a child, he was an avid reader of non fiction books. Aside from his love of children's encyclopedias, his favourite book, "A journey Through Physics" has inspired him about how fascinating the world is. In August 1998, he witnessed a total solar eclipse; a magical moment for a young impressionable child. After this event, he stumbled upon a scientific magazine Наука и техника (Science and technology) with the solar eclipse on the front cover. For the next two years, he followed this and several other popular science magazines and newspapers.

At the age of 11, Alexandar and his family moved to Canada, and he went to High School at Henry Street HS in Whitby, just East of Toronto. At Henry Street, he was part of several extracurricular groups such as Concert band and Jazz band, Ski club and Tennis club, French club, Culture of Peace and school Reach. 

Alexandar was admitted to the Bachelor's of Physics, Co-operative program at the University of Waterloo. Through this program, he spent his work terms working with teams in Canada (AECL, TRIUMF and SNOLab) and Europe (Kiruna and Geneva). After completing his degree, he was admitted to a Master's of Astronomy and Astrophysics at KULeuven, Belgium. In this international environment he met his fiancee, Rossella, and they began a long fruitful relationship spanning multiple countries in Europe. At the completion of his Master's degree, Alexandar spent the summer at Leiden University as a LEAPS summer student, working with Airbus on algorithms to correct for frame smearing for an upcoming ESA Sentinel mission. 

After the LEAPS project he began a Doctorate project, ``Understanding the Computational Challenges for current and Next Generation Radio Telescopes.'' He worked with Prof. Huub Rottgering and Dr. Huib Intema from the Leiden Observatory as well as with Prof. Aske Plaat from the Leiden Institute of advanced computer science. Thanks to an invaluable collaboration with the Dutch National computing centre at SURFsara in Amsterdam, Alexandar was able to develop the tools to efficiently process substantial quantities of LOFAR data, and better undestand the remaining bottlenecks. 

 
 
