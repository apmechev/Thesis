
\makeglossaries

\newglossaryentry{LOFAR}
{
    name=LOFAR,
    description={The LOw Frequency ARray: A large, low-frequency aperture synthesis radio telescope} 
}

\newglossaryentry{LoTSS}
{
    name=LoTSS,
    description={The LOFAR Two-Meter Sky Survey is a whole-sky study of the low-frequency radio sky at 120-168MHz. LoTSS is composed of a broad, Tier 1, survey of the entire sky, as well as deeper tiers targeted at specific fields of interest }
}


\newglossaryentry{Subband}
{
    name=Subband,
    description={Broadband LOFAR observations are stored in separate 'Subbands', splitting the frequency range into several individual files, before storing at the Long Term Archive. Depending on the observation mode, one observation can have 230-480 Subbands. This splitting makes it easier for users to request, download and process a fraction of observation's entire bandwidth}
}



\newglossaryentry{UV plane}
{
    name={UV plane},
    description={Radio Astronomers use the term `UV plane' to refer to the Fourier Transform of the final image. Each baseline of an aperture synthesis telescope corresponds to one sample of the UV plane. The letters U and V refer to the two  orthogonal components of a baseline with respect to an observation's phase center. The $u$ and $v$ vectors are defined in a plane orthogonal to the direction towards the phase center, and are typically in units of wavelength}
}
