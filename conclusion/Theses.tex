1. Building a framework to process massive amounts of data drastically accelerates the rate of scientific discoveries (Ch 2, 3, 5)

2. Integrating scientific processing with industry software not only brings a wide array of features, but ensures a support from a large, expert, community (ch 5)

3. If you automatically test your software, more people will trust it and use it (Ch 7)

4. Understanding how your software performs is best done before you push the performance limits (Ch 4, 6) 

5. If someone with unlimited computing and storage tried to, exactly, reproduce the LOFAR surveys results; they would not be able to.

6. Just because you've written a nice piece of code, doesn't mean anyone will use it.  

7. Traditional education and traditional academia are ripe for disruption. 

8. There exists a critical mass of users below which scientific software withers away and dies.

9. In the long run, learning new tools is less painful than using old tools to do new things. 

10. The number of project ideas you have grows proportionally with the number of skills you acquire.

11.  

12. 
